%% Otros niveles
\NewSubPart{En Otros Niveles Educativos}{}

  %% SdEC - Evaluación
  \WorkEntry{Taller para docentes de Evaluación para el aprendizaje}{Julio 2024}{Universidad de Buenos Aires}{Durante la \textbf{Semana de la Enseñanza de las Ciencias} 2024}

    \sepspace

  %% SdEC - Electrónica
  \WorkEntry{Taller para docentes de Electrónica Aplicada}{Julio 2017 - 2019 y 2022 - 2024}{Universidad de Buenos Aires}{Durante la \textbf{Semana de la Enseñanza de las Ciencias} ediciones 2017 - 2019 y 2022 - 2024}

    \sepspace

  %% SdEC - Didáctica
  \WorkEntry{Taller para docentes de Didáctica de la Programación}{Julio 2018, 2019, y 2022 - 2024}{Universidad de Buenos Aires}{Durante la \textbf{Semana de la Enseñanza de las Ciencias} ediciones 2018, 2019 y 2022 - 2024}

    \sepspace

  %% Cx1D
  \WorkEntry{Taller para estudiantes Científic@s por un día}{2016 - 2019 y 2022 - 2023}{Universidad de Buenos Aires}

    \sepspace

  %% Profesorados 23 - Didáctica
  \WorkEntry{Taller para docentes de Didáctica de la Programación}{Noviembre 2023}{Universidad de Buenos Aires}{XVII Encuentro Internacional de Profesorados de Enseñanza Superior, Media y Primaria en Ciencias Naturales, Matemática y Tecnología}

    \sepspace

\newpage

  %% DOV 23
  \WorkEntry{Taller para estudiantes ``La `Magia' de la Computación:\\Resolviendo desafíos a través de la Programación''}{2023}{Universidad de Buenos Aires}

    \sepspace
  
  %% Profesorados 22 - Electrónica
  \WorkEntry{Taller para docentes de Electrónica Aplicada}{Noviembre 2022}{Universidad de Buenos Aires}{XVI Encuentro Internacional de Profesorados de Enseñanza Superior, Media y Primaria en Ciencias Naturales, Matemática y Tecnología}

    \sepspace

  %% Ludover 21 y 22
  \WorkEntry{Taller de Programación de Videojuegos}{2021 y 2022}{Talleres Uqbar}

    \sepspace

  %% FdL - Didáctica 22
  \WorkEntry{Taller de Didáctica para docentes ``¿Cómo Enseñar a Programar?''}{Mayo 2022}{La Rural}{Durante el \textbf{20° Foro Internacional de Enseñanza de Ciencias y Tecnologías} en la \textbf{Feria del Libro}}

    \sepspace

  %% DOV 17, 19, 20, 21 y 22
  \WorkEntry{Taller para estudiantes ``Aventuras Computacionales:\\Resolviendo Problemas con y sin Computadoras''}{2017 y 2019 - 2022}{Universidad de Buenos Aires}

  \sepspace

  %% Profesorados 19 - Programación y Robótica
  \WorkEntry{Taller para docentes de Programación y Robótica\\en la escuela}{Noviembre 2019}{Universidad de Buenos Aires}{XIII Encuentro Internacional de Profesorados de Enseñanza Superior, Media y Primaria en Ciencias Naturales, Matemática y Tecnología}

    \sepspace

  %% SdEC - AppInventor
  \WorkEntry{Taller para docentes de Programación de Aplicaciones\\Móviles en el Aula}{Julio 2019}{Universidad de Buenos Aires}{Durante la \textbf{Semana de la Enseñanza de las Ciencias} 2019}

    \sepspace

  %% Sadosky A
  \WorkEntry{Docente (equiparado a Ayudante de Segunda)}{Agosto 2018 - Julio 2019}{Universidad de Buenos Aires}{En el taller para estudiantes \textbf{Introducción a la Programación}}

    \sepspace

  %% FdL - Didáctica 19
  \WorkEntry{Taller de Didáctica para docentes ``¿Cómo Enseñar a Programar?''}{Abril 2019}{La Rural}{Durante el \textbf{19° Foro Internacional de Enseñanza de Ciencias y Tecnologías} en la \textbf{Feria del Libro}}

    \sepspace

  %% Profesorados 18 - Electrónica
  \WorkEntry{Taller para docentes de Electrónica Aplicada}{Noviembre 2018}{Universidad de Buenos Aires}{Durante el \textbf{XII Encuentro Internacional de Profesorados de Enseñanza Superior, Media y Primaria en Ciencias Naturales, Matemática y Tecnología}}

    \sepspace

\newpage
    
  %% SdEC - Robótica
  \WorkEntry{Taller para docentes de Programación y Robótica}{Julio 2017 y 2018}{Universidad de Buenos Aires}{Durante la \textbf{Semana de la Enseñanza de las Ciencias} ediciones 2017 y 2018}

    \sepspace

  %% Profesorados 17 - Electrónica
  \WorkEntry{Taller para docentes de Electrónica Aplicada}{Noviembre 2017}{Universidad de Buenos Aires}{Durante el \textbf{XI Encuentro Internacional de Profesorados de Enseñanza Superior, Media y Primaria en Ciencias Naturales y Matemática}}

%%% Experiencia Investigación
%%% ------------------------------------------------------------
\NewPart{Experiencia en Investigación}{}

  % CLEI 2024
  \WorkEntry{2024 \textbf{AelE: a versatile tool for teaching programming and robotics using Arduino
    %\footnote{\url{}}
  }}{}
  {Gonzalo Pablo Fernández, Christian Cossio-Mercado}
  {\textbf{50 Conferencia Latinoamericana de Informática (L CLEI 2024), 12-16 de agosto de 2024, Bahía
  Blanca, Argentina}}

    \sepspace

  % JAR 2024
  \WorkEntry{2024 \textbf{AelE: una herramienta para la enseñanza de programación basada en Arduino
    %\footnote{\url{}}
  }}{}
  {Gonzalo Pablo Fernández, Christian Cossio-Mercado}
  {\textbf{Jornadas Argentinas de Robótica (JAR), 4-7 de junio de 2024, Buenos Aires, Argentina}}

    \sepspace

  % JADiCC 2023
  \WorkEntry{2023 \textbf{PRENDER: Una propuesta didáctico-pedagógica para la enseñanza de las Ciencias de la Computación
    %\footnote{\url{}}
  }}{}
  {Christian Cossio-Mercado, Gonzalo Pablo Fernández}
  {\textbf{Jornadas Argentinas de Didáctica de las Ciencias de la Computación (JADICC), 1-2 de diciembre de 2023, Neuquén, Argentina}}

    \sepspace

  % JAIIO-ex 2023
  \WorkEntry{2023 \textbf{Relevamiento de conocimientos previos de programación en el nivel universitario
    \footnote{\url{https://publicaciones.sadio.org.ar/index.php/EJS/article/view/862/701}}
  }}{}
  {Gonzalo Pablo Fernández, Cecilia Martínez, Pablo E. ``Fidel'' Martínez López}
  {\textbf{Electronic Journal of SADIO (EJS) 23 (2) 2024 pg 150-175}}

    \sepspace

  % JAIIO 2023
  \WorkEntry{2023 \textbf{Relevamiento de conocimientos previos de programación en el nivel universitario
    \footnote{\url{https://publicaciones.sadio.org.ar/index.php/JAIIO/article/view/632/647}}
  }}{}
  {Gonzalo Pablo Fernández, Cecilia Martínez, Pablo E. ``Fidel'' Martínez López}
  {\textbf{Simposio Argentino de Educación en Informática (SAEI) - Jornadas Argentinas de Informática e Investigación Operativa (JAIIO) 2023}}

    \sepspace

  % JADiCC 2022
  \WorkEntry{2022 \textbf{Aprender programación usando bloques y texto en forma simultánea - Un enfoque espiralado
    %\footnote{\url{}}
  }}{}
  {Gonzalo Pablo Fernández, Pablo E. ``Fidel'' Martínez López, Cecilia Martínez}
  {\textbf{Jornadas Argentinas de Didáctica de las Ciencias de la Computación (JADICC) 2022}}

    \sepspace

  % JADiCC 2021
  \WorkEntry{2021 \textbf{Arduino en la Escuela: una herramienta versátil para la enseñanza de programación y robótica
    \footnote{\url{https://jadicc2021.program.ar/wp-content/uploads/2021/10/JADICC2021_paper_56.pdf}}}}{}
  {Gonzalo Pablo Fernández, María Belén Ticona Oquendo, Christian Cossio-Mercado}
  {\textbf{Jornadas Argentinas de Didáctica de las Ciencias de la Computación (JADICC) 2021}}

    \sepspace
\newpage
  % ICPR 2020
  \WorkEntry{2020 \textbf{Attribute classification for the analysis of genuineness of facial expressions
    \footnote{\url{https://www.researchgate.net/publication/350189026_Attribute_classification_for_the_analysis_of_genuineness_of_facial_expressions}}}}{}
  {Gonzalo Fernández, María Elena Buemi, Daniel Germán Acevedo, Pablo Negri}
  {\textbf{International Conference on Pattern Recognition (ICPR) 2020}}

%%% Experiencia Divulgación
%%% ------------------------------------------------------------
\NewPart{Experiencia en Divulgación}{}

  %% Divu
  \WorkEntry{Divulgador Departamental}{2017 - 2018}{Universidad de Buenos Aires}

%% Proyectos
\NewSubPart{Proyectos}{}

  %% Arduino en la Escuela
  \WorkEntry{Arduino en la Escuela}{Desde 2019}{Universidad de Buenos Aires}{Dictado del taller \textbf{Electrónica Aplicada} y desarrollo de la aplicación}

  %% Robótica en la Escuela
  \WorkEntry{Robótica en la Escuela}{2018 - 2019 y desde 2022}{Universidad de Buenos Aires}{Dictado del taller \textbf{Robótica Educativa} y diseñador Blender}

  %% Tecnópolis - Imaginación
  \WorkEntry{Exactas en Tecnópolis}{Durante 2022 y 2023}{Tecnópolis}{Armado de actividad educativa}

  %% Robótica en el Barrio
  \WorkEntry{Robótica en el Barrio}{Marzo - Abril 2019}{Independiente}{Dictado del taller \textbf{Robótica y Electrónica}}

  %% Talleres ComCom
  \WorkEntry{Talleres ComCom para estudiantes de Computación}{2018 - 2019}{Independiente}{Armado y dictado de distintos talleres}

  %% Talleres CBC
  \WorkEntry{Talleres para estudiantes del CBC de Computación}{2018 - 2019}{Programa de Tutorías CBC}{Dictado de los talleres de \textbf{Programación Musical} y de \textbf{Robótica Educativa}}

  %% Bacterminador
  \WorkEntry{Bacterminador}{2017 - 2019}{Universidad de Buenos Aires}{Programador Godot Engine}

%% Actividades
\NewSubPart{Actividades}{}

  %% 23/8/24 Feria CIDAC-UBA

  %% Feria Mujer y Niña en la Ciencia
  \WorkEntry{Feria del Día de la Mujer y la Niña en la Ciencia}{2023 y 2024}{Ciudad Universitaria}

  %% 11/11/23 Feria Escuela Cullen

  %% FdL
  \WorkEntry{Feria del Libro}{Mayo 2017 - 2019 y 2022 - 2024}{La Rural}

  %% Feria de Profesiones
  \WorkEntry{Feria de Profesiones ``¿Para qué seguir\\estudiando?''}{2017 - 2019 y 2022 - 2023}{Colegio Técnico de la UBA - Villa Lugano}

  %% NdlM
  \WorkEntry{La Noche de la Ciencia}{Noviembre 2016 - 2019 y 2022 - 2023}{Universidad de Buenos Aires}{Durante la \textbf{Noche de los Museos}}

  %% Feria +Ciencia +Futuro (San Martín)
  \WorkEntry{Feria +Ciencia +Futuro}{Octubre 2023}{San Martín - Buenos Aires}

  %% SdC
  \WorkEntry{Organización de la Semana de la Computación}{2017 - 2019 y 2022 - 2023}{Universidad de Buenos Aires}

  %% SdC
  \WorkEntry{Semana de la Computación desde Casa}{2020 y 2021}{Universidad de Buenos Aires}

  %% Plaza Ciencia
  \WorkEntry{Plaza Ciencia}{Octubre 2017, 2018 y 2019}{Plaza San Justo - La Matanza}

  %% Curioseando
  \WorkEntry{Feria de ciencia y arte Curioseando}{Septiembre 2019}{Escuela N$^{\circ}$ 22 Remedios de Escalada de San Martín}

  %% CP 2019
  \WorkEntry{Ciencia Paliza 2}{Junio 2019}{Parque Rivadavia}

  %% MateArte
  \WorkEntry{Feria MateArte Científico}{Ocutbre 2018}{Escuela N°15 Provincia de Santa Fé}{Dictado del taller \textbf{La Computación está en todas partes}}

  %% Expotécnica
  \WorkEntry{Feria Expotécnica}{Octubre 2018}{Escuela Técnica N°3 Roberto Arlt - Tortuguitas}{Dictado del taller de \textbf{Robótica Educativa}}

  %% VocAr
  \WorkEntry{Taller Aventuras Computacionales}{Agosto 2018 y Octubre 2019}{Espacio VocAr - Tecnópolis}

  %% Tecnópolis
  \WorkEntry{Taller La Computación está en todas partes}{Septiembre 2018}{Tecnópolis}

  %% Ciencia Palooza
  \WorkEntry{Un Festival de Ciencia}{Marzo 2018}{Centro Cultural Konex}

  %% C3
  \WorkEntry{Taller Hoy te convertís en Superhéroe}{Julio 2017}{Centro Cultural de la Ciencia}

  %% Penal
  \WorkEntry{Feria ``¿Qué onda con la Ciencia?''}{Julio 2017}{UBA XXII - Penal de Devoto}

  %% SdC
  \WorkEntry{Colaboración en la Semana de la Computación}{Junio 2014, 2015 y 2016}{Universidad de Buenos Aires}

%%% Experiencia Laboral
%%% ------------------------------------------------------------
\NewPart{Experiencia Laboral}{}

  %% ECI-bot
  \WorkEntry{Desarrollo Freelance en python}{Junio 2021}{20 Horas}

    \sepspace

  %% GPFTW
  \WorkEntry{Desarrollo Freelance en C++}{Diciembre 2018}{20 Horas}

    \sepspace

  %% +Simple
  \WorkEntry{Desarrollo Freelance en Android Studio}{Junio 2016}{20 Horas}

    \sepspace

  %% Carrito
  \WorkEntry{Desarrollo Freelance en Unity 3D}{Diciembre 2013}{40 Horas}

  \sepspace


%%% Formación Pedagógica
%%% ------------------------------------------------------------
\NewPart{Formación Pedagógica}{}

  %% JADiCC 2023
  \WorkEntry{Jornadas Argentinas de Didáctica de las Ciencias de la Computación}{2023}{Universidad Nacional del Comahue}

    \sepspace

  %% JADiCC 2022
  \WorkEntry{Jornadas Argentinas de Didáctica de las Ciencias de la Computación}{2022}{Universidad Nacional del Nordeste}

    \sepspace

  %% JADiCC 2021
  \WorkEntry{Jornadas Argentinas de Didáctica de las Ciencias de la Computación}{2021}{Virtual}

    \sepspace

  %% ADICRA 2019
  \WorkEntry{Foro de la Asociación de Docentes de Informática\\y Computación de la República Argentina (ADICRA)}{Noviembre 2019}{}

    \sepspace

  %% JADiPro 2019
  \WorkEntry{Segundas Jornadas Argentinas de Didáctica de la Programación}{Junio 2019}{Universidad de Córdoba}

    \sepspace

  %% Capacitación Sadosky B2
  \WorkEntry{Capacitación para el dictado del curso ``La Programación\\y su Didáctica 2''}{Febrero 2019}{Fundación Sadosky}{40 Horas}

    \sepspace

  %% Taller Diseño 2018
  \WorkEntry{Diseño de Experiencias y Materialización de\\Divulgación Científica}{Agosto - Septiembre 2018}{Universidad de Buenos Aires}{15 Horas}

    \sepspace

  %% JADiPro 2018
  \WorkEntry{Primeras Jornadas Argentinas de Didáctica de la Programación}{Junio 2018}{Universidad de Quilmes}

    \sepspace

  %% Taller Talleristas 2018
  \WorkEntry{Seminario de didáctica de las ciencias: Aportes para las\\intervenciones de extensionistas}{Abril - Mayo 2018}{Universidad de Buenos Aires}{12 Horas}

    \sepspace

  %% Capacitación Sadosky B1
  \WorkEntry{Capacitación para el dictado del curso ``La Programación\\y su Didáctica''}{Julio 2017}{Fundación Sadosky}{40 Horas}

    \sepspace

  %% Taller Talleristas 2017
  \WorkEntry{Trayecto de formación profesional para docentes universitarios\\en el marco de la preparación de talleres para las\\Semanas de las Ciencias}{Abril - Junio 2017}{Universidad de Buenos Aires}{20 Horas}
\newpage
%%% Habilidades
%%% ------------------------------------------------------------
\NewPart{Habilidades}{}

\NewSubPart{Idiomas}{}

\SkillsEntry{Español}{Nativo}

\SkillsEntry{Inglés}{Intermedio}

\NewSubPart{Computación}{}

\SkillsEntry{Software}{\textsc{c}, \textsc{c++}, \textsc{java}, \textsc{python},
%% ply, matplotlib, scikit-learn, keras
\textsc{javascript},}
\SkillsEntry{}{\textsc{haskell}, \textsc{smalltalk}, \textsc{gobstones},}
\SkillsEntry{}{\textsc{blender}, \textsc{godot}, \textsc{unity3d},}
\SkillsEntry{}{\textsc{verilog}, \textsc{assembler}, \textsc{matlab}}

\sepspace

\SkillsEntry{Otros}{\textsc{linux}, \LaTeX, \textsc{git}, \textsc{html}, \textsc{sql},}
\SkillsEntry{}{\textsc{visual studio}, \textsc{eclipse}, \textsc{arduino},}
\SkillsEntry{}{\textsc{android studio}, \textsc{sonic pi}
%%, \textsc{tomcat}, \textsc{nginx}, \textsc{gnuplot}
}


%%% References
%%% ------------------------------------------------------------
%\NewPart{References}{}
%Available upon request