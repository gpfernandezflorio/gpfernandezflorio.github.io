
% Datos personales
\newcommand{\nombresDelAspirante}{Gonzalo Pablo}
\newcommand{\apellidoDelAspirante}{Fernández}
\newcommand{\numeroDocumentoDeIdentidadDelAspirante}{35.971.989}
\newcommand{\numeroDiaNacimientoDelAspirante}{24}
\newcommand{\numeroMesNacimientoDelAspirante}{04}
\newcommand{\anioNacimientoDelAspirante}{1991}
\newcommand{\telefonoDelDomicilioDelAspirante}{+54-11-15-5317-1918}
\newcommand{\emailDelAspirante}{gpfernandez@dc.uba.ar}
\newcommand{\emailAlternativoDelAspirante}{}
% Completar únicamente si lo tuviera. Si no, dejarlo en blanco
\newcommand{\numeroDeLegajoDelAspirante}{191897}

% Descomentar la línea que corresponda
% \newcommand{\JTP}{} % Jefe/a de trabajos prácticos
\newcommand{\AYp}{} % Ayudante de 1°
% \newcommand{\AYs}{} % Ayudante de 2°

% Descomentar la línea que corresponda (para Ay2 es siempre Parcial)
\newcommand{\Parcial}{} % Parcial
% \newcommand{\Semi}{} % SemiExclusiva
% \newcommand{\Exclu}{} % Exclusiva

% Área a concursar "Algoritmos", "Sistemas" o "Teoría"
\newcommand{\area}{Pensamiento Computacional}

% Completar si te tomaste alguna licencia
% Si estás concursando por primera vez, dejalo como está
\newcommand{\licencias}{
  \begin{itemize}
    \item 01/08/2023 - 29/02/2024
  \end{itemize}
}

% Definir si querés que tengan en cuenta tu desempeño docente durante la pandemia (si no querés comentá la definición del comando)
% Si estás concursando por primera vez, dejalo como está
\newcommand{\considerameLaPandemia}{}

% Si tenés una imagen con tu firma, descomentá la siguiente línea para que se agregue en cada carilla:
\newcommand{\firma}{firma.jpg}
% Si la imagen es muy chica o muy grande, modificá a continuación los valores de escala:
\newcommand{\escalaFirmaPrincipal}{0.1}
\newcommand{\escalaFirmaCadaCarilla}{0.06}
