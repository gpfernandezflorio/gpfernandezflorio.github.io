% Divu
\begin{itemize}[leftmargin=0.5cm]
\WorkEntry{\textbf{Divulgador Departamental}}
{FCEyN - UBA}
{Durante 2017 y 2018}{}
\end{itemize}

\begin{enumerate}[leftmargin=0.8cm]

  \item[a)]{Proyectos de Extensión actuales y anteriores

    \begin{itemize}[leftmargin=0.2cm]

      \item[i)]{realizados en el ámbito de las Universidades Nacionales.

        \begin{itemize}[leftmargin=0.2cm]

          %% AelE + Unipe
          % \WorkEntry{Participación en el proyecto \textbf{?}}
          % {Universidad Pedagógica Nacional}
          % {Desde el segundo cuatrimestre de 2024.}

          %% Tecnópolis - Imaginación
          \WorkEntry{Participación en el proyecto \textbf{Exactas en Tecnópolis}}
          {FCEyN - UBA}
          {Entre 2022 y 2023.}
          %{Armado de actividad educativa}

          % Arduino en la Escuela
          \WorkEntry{Participación en el proyecto \textbf{Arduino en la Escuela}\footnote{\url{aele.dc.uba.ar}}}
          {FCEyN - UBA}
          {Desde 2018.}
          %{Desarrollé una nueva herramienta de programación por bloques para Arduino para ser
            %utilizada en los talleres de electrónica.}

          % Robótica en la Escuela
          \WorkEntry{Participación en el proyecto \textbf{Robótica en la Escuela}\footnote{A cargo del
            Departamento de Computación de la facultad, como parte del programa \textbf{UBANEX}. Consiste
            en un proyecto para promover la enseñanza de la programación y la robótica en las escuelas,
            mediante el uso de un simulador online (\url{http://www.roboticaenlaescuela.dc.uba.ar/}).}}
          {FCEyN - UBA}
          {Entre 2018 y 2019 y desde 2022}
          %{Diseñé en Blender la nueva versión del robot que utilizaremos en los talleres y cuyas
            %especificaciones se compartirán a las escuelas para que puedan armar su propio robot.
            %También armé nuevos mapas para ser utilizados en los talleres.}

          % Robótica en el Barrio
          \WorkEntry{Participación en el proyecto \textbf{Robótica en el Barrio}\footnote{Taller de robótica y
            electrónica con Arduino para chicos y chicas de primaria y secundaria dictado en un centro comunitario en Longchamps.}}
          {FCEyN - UBA}
          {Entre Marzo y Abril 2019. 4 clases de 4 horas.}
          %{}

          % Talleres ComCom
          \WorkEntry{Participación en los \textbf{Talleres ComCom}\footnote{Talleres para estudiantes de la carrera
            de computación y otras carreras de la facultad.}}
          {FCEyN - UBA}
          {Entre 2018 y 2019.}
          %{Armado y dictado de distintos talleres. \\
          %Hasta ahora sólo pudimos dictar el \textbf{Taller Introductorio de Linux}, de 2 clases de 2 horas, pero tenemos otros armados.}

          % Talleres CBC
          \WorkEntry{Participación en los \textbf{Talleres CBC}\footnote{Talleres para estudiantes cursando el CBC de
            la carrera de computación, organizado junto al Programa de Tutorías CBC de la carrera de computación.}}
          {FCEyN - UBA}
          {Durante 2018.}
          %{Armado y dictado de distintos talleres. \\
          %Dicté los talleres de \textbf{Programación y Robótica} y de \textbf{Programación Musical}.}

          % Bacterminador
          \WorkEntry{Participación en el proyecto \textbf{Bacterminador}\footnote{A cargo del Departamento
            de Química Orgánica de la facultad, como parte del prograba \textbf{UBANEX}. Consiste en un juego
            pensado para ser utilizado por docentes de escuela primaria para enseñar conceptos básicos de higiene
            y microbiología de alimentos. Detalles del proyecto: \url{http://bacterminador.com.ar/sobre-el-proyecto/}}}
          {FCEyN - UBA}
          {Entre 2017 y 2019.}
          %{Además de algunas mejoras menores, rehice la aplicación en el motor gráfico Godot para
            %que sea más fácil de mantener que la versión anterior (desarrollada en Construct 2) y
            %para poder exportarla a diferentes plataformas, entre ellas, web, Windows, Linux, Mac y Android.}

        \end{itemize}
      }

    \end{itemize}
  }

  \item[b)]{Actividades

    \begin{itemize}[leftmargin=0.2cm]

      \item[i)]{de divulgación científica.

        \begin{itemize}[leftmargin=0.2cm]

          % FdL mayo 2024
          \WorkEntry{Participación en la \textbf{Feria del Libro} 2024 - Zona Explora}
          {}{12 de Mayo de 2024, de 12:30hs a 17:30hs.}

          %% Feria Mujer y Niña en la Ciencia
          \WorkEntry{Participación en la \textbf{Feria del Día de la Mujer y la Niña en la Ciencia} 2024}{19 de Febrero de 2024, de 9hs a 13hs.}{Ciudad Universitaria}

          %% 11/11/23 Feria Escuela Cullen

          % NdlM noviembre 2023
          \WorkEntry{Participación en \textbf{La Noche de la Ciencia} 2023}
          {La Noche de los Museos}
          {3 de Noviembre de 2023, de 19hs a 0hs.}

          % Feria de Profesiones octubre 2023
          \WorkEntry{Participación en la jornada \textbf{¿Para qué seguir estudiando?}\footnote{Feria
            de profesiones en la \textbf{Escuela Técnica de la UBA} en Villa Lugano, como parte del
            programa de tutorías \textbf{Universitarios por más Universitarios} (\url{http://www.uba.ar/contenido/448}).}\ 2023}
          {Escuela Técnica - UBA (Sede Lugano)}
          {28 de Octubre de 2023, de 9:30hs a 13hs.}

          %% Feria +Ciencia +Futuro (San Martín)
          \WorkEntry{Participación en la \textbf{Feria +Ciencia +Futuro}}{7 de Octubre de 2023, de 11hs a 17hs.}{San Martín - Buenos Aires}

          % SdC septiembre 2023
          \WorkEntry{\textbf{Organizador General} en la \textbf{Semana de la Computación y las Ciencias de Datos} 2023}
          {}{5, 6 y 7 de Septiembre de 2023}

          % FdL mayo 2023
          \WorkEntry{Participación en la \textbf{Feria del Libro} 2023 - Zona Explora}
          {}{14 de Mayo de 2023, de 13hs a 21hs.}

          %% Feria Mujer y Niña en la Ciencia
          \WorkEntry{Participación en la \textbf{Feria del Día de la Mujer y la Niña en la Ciencia} 2023}{16 de Febrero de 2023, de 9hs a 13hs.}{Ciudad Universitaria}

          % NdlM octubre 2022
          \WorkEntry{Participación en \textbf{La Noche de la Ciencia} 2022}
          {La Noche de los Museos}
          {22 de Octubre de 2022, de 20hs a 3hs.}

          % Feria de Profesiones octubre 2022
          \WorkEntry{Participación en la jornada \textbf{¿Para qué seguir estudiando?} 2022}
          {Escuela Técnica - UBA (Sede Lugano)}
          {15 de Octubre de 2022, de 9hs a 13hs.}
          %{\emph{Expositor} en las distintas estaciones demostrativas.}

          % SdC ocutbre 2022
          \WorkEntry{\textbf{Organizador General} en la \textbf{Semana de la Computación y las Ciencias de Datos} 2022}
          {}{4, 5 y 6 de Octubre de 2022}

          % FdL mayo 2022
          \WorkEntry{Participación en la \textbf{Feria del Libro} 2022 - Zona Explora}
          {}{15 de Mayo de 2022, de 12hs a 21hs.}
          %{\emph{Expositor} en varias estaciones demostrativas.}

          % NdlM noviembre 2019
          \WorkEntry{Participación en \textbf{La Noche de la Ciencia} 2019}
          {La Noche de los Museos}
          {2 de Noviembre de 2019, de 20hs a 3hs.}
          %{\emph{Animador}. \\
          %También participé en el armado de la Sala de Escape.}

          % Vocar octubre 2019
          \WorkEntry{Participación en la actividad \textbf{Aventuras Computacionales}\footnote{Dos instancias de un taller
            para estudiantes de escuelas primarias en la que se les enseña búsqueda binaria a través de un juego.}}
          {Tecnópolis, Espacio VocAr del Conicet}
          {4 de Octubre de 2019 de 11hs a 15:30hs.}
          %{Colaboré en el armado de la clase.}

          % Plaza Ciencia octubre 2019
          \WorkEntry{Participación en la jornada \textbf{Plaza Ciencia}\footnote{
            \url{http://www.lamatanza.gov.ar/ciencia/plaza-ciencia}}\ 2019}
          {Plaza San Justo, La Matanza}
          {2 de Octubre de 2019, de 13hs a 17hs y 3 de octubre de 2018 de 10hs a 17hs.}
          %{\emph{Expositor} en las distintas estaciones demostrativas.}

          % Feria Curioseando septiembre 2019
          \WorkEntry{Participación en la feria de ciencia y arte \textbf{Curioseando}}
          {Escuela N$^{\circ}$ 22 Remedios de Escalada de San Martín}
          {28 de Septiembre de 2019, de 16hs a 19hs.}
          %{\emph{Expositor} en las distintas estaciones demostrativas.}

          % Feria de Profesiones septiembre 2019
          \WorkEntry{Participación en la jornada \textbf{¿Para qué seguir estudiando?} 2019}
          {Escuela Técnica - UBA (Sede Lugano)}
          {28 de Septiembre de 2019, de 9hs a 14hs.}
          %{\emph{Expositor} en las distintas estaciones demostrativas.}

          % SdQ septiembre 2019
          \WorkEntry{Participación en la \textbf{Semana de la Química} 2019}
          {}{Septiembre de 2019.}
          %{Desarrollé el software para la Sala de Escape.}

          % SdC septiembre 2019
          \WorkEntry{\textbf{Organizador General} en la \textbf{Semana de la Computación} 2019}
          {}{10, 11 y 12 de Junio de 2019, de 9hs a 16hs, durante toda la jornada.}
          %{\emph{Docente} en varias instancias del \textbf{Taller de Robótica}, del \textbf{Taller de Programación
            %Musical} y del \textbf{Taller de Animaciones y Juegos} (algunos a cargo y otros como colaborador). \\
            %\emph{Expositor} en varias estaciones demostrativas y en la Sala de Escape. \\
            %También participé en el armado de la Sala de Escape y del nuevo Taller de Animaciones y Juegos.}

          %% CP 2 Junio 2019
          \WorkEntry{Participación en \textbf{Ciencia Paliza 2}}
          {Parque Rivadavia}
          {1 de Junio de 2019, de 15hs a 18hs.}
          %{\emph{Expositor} en las distintas estaciones demostrativas.}

          % SdCT mayo 2019
          \WorkEntry{Participación en la \textbf{Semana de las Ciencias de la Tierra} 2019}
          {}{Mayo de 2019.}
          %{Desarrollé el software para la Sala de Escape.}

          % FdL mayo 2019
          \WorkEntry{Participación en la \textbf{Feria del Libro} 2019 - Zona Explora}
          {}{12 de Mayo de 2019, de 12hs a 17hs.}
          %{\emph{Colaborador} en el Taller de \textbf{Introducción a la Programación}.}

          % SdM abril 2019
          \WorkEntry{Participación en la \textbf{Semana de la Matemática} 2019}
          {}{Abril de 2019.}
          %{Desarrollé el software para la Sala de Escape.}

          % NdlM noviembre 2018
          \WorkEntry{Participación en \textbf{La Noche de la Ciencia} 2018}
          {La Noche de los Museos}
          {10 de Noviembre de 2018, de 20hs a 3hs, durante toda la jornada.}
          %{\emph{Animador}. \\
          %También participé en el armado de la Sala de Escape.}

          % Feria MateArte Científico octubre 2018
          \WorkEntry{Participación en la feria \textbf{Mate Arte Científico}}
          {Escuela N$^{\circ}$ 15 Provincia de Santa Fé}
          {27 de Octubre de 2018, de 15hs a 18:30hs.}
          %{\emph{Colaborador} en las instancias del taller de programación.}

          % Plaza Ciencia octubre 2018
          \WorkEntry{Participación en la jornada \textbf{Plaza Ciencia} 2018}
          {Plaza San Justo, La Matanza}
          {1 de Octubre de 2018, de 13hs a 17hs y 6 de octubre de 2018 de 10hs a 15hs.}
          %{\emph{Expositor} en las distintas estaciones demostrativas.}

          % Feria de Profesiones septiembre 2018
          \WorkEntry{Participación en la jornada \textbf{¿Para qué seguir estudiando?} 2018}
          {Escuela Técnica - UBA (Sede Lugano)}
          {29 de Septiembre de 2018, de 9hs a 14hs.}
          %{\emph{Expositor} en las distintas estaciones demostrativas.}

          % Vocar agosto 2018
          \WorkEntry{Participación en la actividad \textbf{Aventuras Computacionales}}
          {Tecnópolis, Espacio VocAr del Conicet}
          {16 de Agosto de 2018, de 14hs a 15:30hs.}
          %{Colaboré en el armado de la clase.}

          % SdC junio 2018
          \WorkEntry{\textbf{Organizador} en la \textbf{Semana de la Computación} 2018}
          {}{12, 13 y 14 de Junio de 2018, de 9hs a 16hs, durante toda la jornada.}
          %{\emph{Docente} en varias instancias de los talleres de Robótica y de Programación
            %Musical (algunos a cargo y otros como colaborador).}

          % FdL mayo 2018
          \WorkEntry{Participación en la \textbf{Feria del Libro} 2018 - Zona Explora}
          {}{7 de Mayo de 2017, de 12:30hs a 17:30hs.}
          %{\emph{Expositor} en el stand de Inteligencia Artificial. \\
          %También colaboré con el armado de material y en el desarrollo de nuevos stands.}

          % Ciencia Palooza marzo 2018
          \WorkEntry{Participación en la jornada \textbf{Un Festival de Ciencia (ex Ciencia Palooza)}\footnote{
            Festival organizado por Expedición Ciencia (\url{http://expedicionciencia.org.ar/}). Detalles del
            evento: \url{http://www.redciteco.org/actividades/eventos/20-cienciapalooza}}}
          {Centro Cultural Konex}
          {29 de Marzo de de 2018, de 18hs a 2hs.}
          %{\emph{Expositor} en el stand de Inteligencia Artificial. \\
          %También colaboré con el armado de material.}

          % NdlM noviembre 2017
          \WorkEntry{Participación en \textbf{La Noche de la Ciencia} 2017}
          {La Noche de los Museos}
          {4 de Noviembre de 2017, de 19hs a 3hs, durante toda la jornada.}
          %{\emph{Expositor} en las distintas estaciones demostrativas.}

          % Plaza Ciencia octubre 2017
          \WorkEntry{Participación en la jornada \textbf{Plaza Ciencia} 2017}
          {Plaza San Justo, La Matanza}
          {2 de Octubre de 2017, de 10hs a 17hs y 4 de octubre de 2017 de 10hs a 13hs.}
          %{\emph{Expositor} en las distintas estaciones demostrativas.}

          % Feria de Profesiones septiembre 2017
          \WorkEntry{Participación en la jornada \textbf{¿Para qué seguir estudiando?} 2017}
          {Escuela Técnica - UBA (Sede Lugano)}
          {23 de septiembre de 2017, de 9hs a 14hs.}
          %{\emph{Expositor} en las distintas estaciones demostrativas.}

          % C3 julio 2017
          \WorkEntry{Participación en el taller \textbf{Hoy te Convertís en Superhéroe}}
          {Centro Cultural de la Ciencia}
          {23 y 30 de Julio de 2017, de 16hs a 18hs.}
          %{\emph{Expositor} en las distintas estaciones demostrativas. \\
          %También colaboré con el armado de material.}

          % UBA XXII julio 2017
          \WorkEntry{Participación en la feria de ciencias \textbf{¿Qué onda con la Ciencia?}\footnote{Como
            parte del programa \textbf{UBA XXII}.}}
          {Penal de Devoto}
          {7 de Julio de 2017, de 9hs a 15hs.}
          %{\emph{Expositor} en las distintas estaciones demostrativas.}

          % SdC junio 2017
          \WorkEntry{\textbf{Organizador} en la \textbf{Semana de la Computación} 2017}
          {}{13, 14 y 15 de Junio de 2017, de 9hs a 16hs, durante toda la jornada.}
          %{\emph{Expositor} en el stand de Búsqueda Binaria. \\
          %\emph{Docente} en cuatro instancias del taller de Programación Musical (dos a cargo y dos como colaborador). \\
          %También desarrollé dos nuevos stands (\textbf{Piedra, papel o tijeras} y \textbf{Animovimiento}).}

          % FdL mayo 2017
          \WorkEntry{Participación en la \textbf{Feria del Libro} 2017 - Zona Explora}
          {}{7 de Mayo de 2017, de 12:30hs a 17:30hs.}
          %{\emph{Expositor} en el stand de Búsqueda Binaria. \\
          %También desarrollé el nuevo stand \textbf{Apocalipsis Zombi} basado en la actividad \textbf{Científicos por un día 2016}.}

          % NdlM octubre 2016
          \WorkEntry{Participación en \textbf{La Noche de la Ciencia} 2016}
          {La Noche de los Museos}
          {29 de Octubre de 2016, durante toda la jornada.}
          %{\emph{Expositor} en los stands \textbf{Generador de Anagramas} y \textbf{Morphing de Caras}.}

          % SdC 2016
          \WorkEntry{Participación en la \textbf{Semana de la Computación} 2016}
          {}{28, 29 y 30 de Junio de 2016, los tres días durante toda la jornada.}
          %{\emph{Expositor} en el stand de Búsqueda Binaria. \\
          %También desarrollé en python la aplicación utilizada en el stand y actualicé el juego de las tarjetas. \\
          %Además, asistí en el taller de robótica el miércoles a la mañana.}

          % SdC 2015
          \WorkEntry{Participación en la \textbf{Semana de la Computación} 2015}
          {}{16, 17 y 18 de Junio de 2015, los tres días durante toda la jornada.}
          %{\emph{Colaborador externo} en el stand \textbf{Búsqueda Binaria}. \\
          %También colaboré con el material utilizado en el stand.}

          % SdC 2014
          \WorkEntry{Participación en la \textbf{Semana de la Computación} 2014}
          {}{17, 18 y 19 de Junio de 2014, los tres días durante toda la jornada.}
          %{\emph{Colaborador en estación demostrativa} en el stand \textbf{Búsqueda Binaria}. \\
          %También colaboré con el material utilizado en el stand.}

        \end{itemize}

      }

      \item[ii)]{de articulación con otros niveles educativos.

        \begin{itemize}[leftmargin=0.2cm]

          % Cx1D diciembre 2023
          \WorkEntry{Participación en la jornada \textbf{Científic@s por un Día}\footnote{
            \url{http://www.fcen.uba.ar/dov/cientificos_por_un_dia/cientificos_por_un_dia.htm}}\ 2023}
          {}{14 de Diciembre de 2022, de 9hs a 18hs.}

          % Taller DOV Junio 2023
          \WorkEntry{Participación en el taller \textbf{La `Magia' de la Computación: Resolviendo desafíos a través de la Programación}}
          {}{Del 1 de Junio al 6 de Julio de 2023, 5 clases (la del 29 de junio se pasó al 6 de julio) de 14hs a 17hs. 15 horas.}

          % Cx1D diciembre 2022
          \WorkEntry{Participación en la jornada \textbf{Científic@s por un Día} 2022}
          {}{15 de Diciembre de 2022, de 9hs a 18hs.}

          % Taller DOV Agosto-Septiembre 2022
          \WorkEntry{Participación en el taller \textbf{Aventuras Computacionales: Resolviendo Problemas con y sin
            Computadoras}}
          {}{Del 11 de Agosto al 15 de Septiembre de 2022, 6 clases de 14hs a 17hs. 18 horas.}

          % Taller DOV junio 2021
          \WorkEntry{Participación en el taller \textbf{Aventuras Computacionales: Resolviendo Problemas con y sin
            Computadoras}\footnote{Como parte del programa \textbf{Talleres de Ciencia} coordinado por la
            \textbf{Dirección de Orientación Vocacional} de la facultad.} (edición virtual)}
          {}{Del 3 al 24 de Junio de 2021, 4 clases de 14hs a 16hs. 8 horas.}

          % Taller DOV octubre 2020
          \WorkEntry{Participación en el taller \textbf{Aventuras Computacionales: Resolviendo Problemas con y sin
            Computadoras} (edición virtual)}
          {}{Del 13 de Octubre al 3 de Noviembre de 2020, 4 clases de 14hs a 16hs. 8 horas.}

          % Cx1D diciembre 2019
          \WorkEntry{Participación en la jornada \textbf{Científic@s por un Día} 2019}
          {}{17 de Diciembre de 2019, de 8:30hs a 17hs.}
          %{\emph{Docente}. Participé tanto en el dictado como en el armado de la clase.}

          % EVE octubre 2019
          \WorkEntry{Participación en el \textbf{Taller de Electrónica Aplicada}, como parte del programa
            \textbf{La Escuela Viene a Exactas}\footnote{
            \url{https://exactas.uba.ar/extension/comunicacion-publica-de-la-ciencia/exactas-va-a-la-escuela/}}}
          {}{24 de Octubre de 2019, de 9hs a 12hs.}
          %{\emph{Docente}. Participé tanto en el dictado como en el armado de la clase.}

          % Taller DOV mayo 2019
          \WorkEntry{Participación en el taller \textbf{Aventuras Computacionales: Resolviendo Problemas con y sin
            Computadoras}}
          {}{Del 15 de Mayo al 26 de Junio de 2019, 8 clases de 14hs a 17hs. 24 horas.}
          %{\emph{Docente} en algunas clases y \emph{colaborador} en otras.}

          % Cx1D diciembre 2018
          \WorkEntry{Participación en la jornada \textbf{Científic@s por un Día} 2018}
          {}{14 de Diciembre de 2018, de 8:30hs a 17hs.}
          %{\emph{Docente}. Participé tanto en el dictado como en el armado de la clase.}

          % EVE agosto 2018
          \WorkEntry{Participación en el \textbf{Taller de Programación y Robótica}, como parte del programa
            \textbf{La Escuela Viene a Exactas}\footnote{
            \url{https://exactas.uba.ar/extension/comunicacion-publica-de-la-ciencia/exactas-va-a-la-escuela/}}}
          {}{10 de Agosto de 2018, de 10hs a 12hs.}
          %{\emph{Colaborador}. Participé en el armado de la clase.}

          % Cx1D diciembre 2017
          \WorkEntry{Participación en la jornada \textbf{Científic@s por un Día} 2017}
          {}{14 de Diciembre de 2017, de 9hs a 18hs.}
          %{\emph{Docente}. Participé tanto en el dictado como en el armado de la clase.}

          % EVE octubre 2017
          \WorkEntry{Participación en el \textbf{Taller de Programación y Robótica}, como parte del programa
          \textbf{La Escuela Viene a Exactas}}
          {}{20 de Octubre de 2017, de 9hs a 11hs.}
          %{\emph{Docente}. Participé tanto en el dictado como en el armado de la clase.}

          % Taller DOV septiembre 2017
          \WorkEntry{Participación en el taller \textbf{Aventuras Computacionales: Resolviendo Problemas con y sin
            Computadoras}}
          {}{Del 22 de Septiembre al 3 de Noviembre de 2017, 7 clases de 14hs a 17hs. 21 horas.}
          %{\emph{Docente} en todas las clases. Participé tanto en el dictado como en el armado de cada clase.}

          % EVE septiembre 2017
          \WorkEntry{Participación en el \textbf{Taller de Programación y Robótica}, como parte del programa
            \textbf{La Escuela Viene a Exactas}}
          {}{4 de Septiembre de 2017, de 14hs a 16hs.}
          %{\emph{Docente}. Participé tanto en el dictado como en el armado de la clase.}

          % Cx1D diciembre 2016
          \WorkEntry{Participación en la jornada \textbf{Científic@s por un Día} 2016}
          {}{7 de Diciembre de 2016, de 14hs a 18hs.}
          %{\emph{Colaborador}.}

        \end{itemize}

      }

    \end{itemize}
  }

  \item[c)]{Publicaciones
    \\ No corresponde
  }

  \item[d)]{Presentaciones de proyectos de extensión en congresos, jornadas y otros encuentros de la especialidad.
    \\ No corresponde
  }

  \item[e)]{Otras actividades de extensión no contempladas en los puntos anteriores.
    \\ No corresponde
  }

\end{enumerate}
