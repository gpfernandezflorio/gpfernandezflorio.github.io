\begin{enumerate}[leftmargin=0.8cm]

  \item[a)]{Trabajos Publicados
    \begin{itemize}[leftmargin=0.2cm]

      % JADICC 2024 Modelos de Cómputo
      \WorkEntry{2024 \textbf{Reflexiones sobre la naturaleza de la Computación
        %\footnote{\url{???}}
      }}
      {Gonzalo Pablo Fernández, Pablo E. ``Fidel'' Martínez López, Alejandro Artopoulos, Alejandra Lliteras}
      {\textbf{Jornadas Argentinas de Didáctica de las Ciencias de la Computación (JADICC), 23-25 de octubre de 2024, Río Cuarto, Argentina}}

      % JADICC 2024 Meta
      \WorkEntry{2024 \textbf{¿Cómo llevar lo que sabemos de Metacognición a las clases de programación?
        %\footnote{\url{???}}
      }}
      {Christian Cossio-Mercado, Gonzalo Pablo Fernández, Gastón Pérez}
      {\textbf{Jornadas Argentinas de Didáctica de las Ciencias de la Computación (JADICC), 23-25 de octubre de 2024, Río Cuarto, Argentina}}

      % JADICC 2024 DOV
      \WorkEntry{2024 \textbf{¿Es posible enseñar los fundamentos de la programación sólo con Arduino?: Análisis de un curso introductorio para estudiantes de secundaria usando bloques
        %\footnote{\url{???}}
      }}
      {Gonzalo Pablo Fernández, Christian Cossio-Mercado}
      {\textbf{Jornadas Argentinas de Didáctica de las Ciencias de la Computación (JADICC), 23-25 de octubre de 2024, Río Cuarto, Argentina}}

      % JADICC 2024 UBA XXII
      \WorkEntry{2024 \textbf{Iniciación a la programación en contextos de encierro con Arduino
        %\footnote{\url{???}}
      }}
      {Daniela Macario Cabral, Christian Cossio-Mercado, Gonzalo Pablo Fernández}
      {\textbf{Jornadas Argentinas de Didáctica de las Ciencias de la Computación (JADICC), 23-25 de octubre de 2024, Río Cuarto, Argentina}}

      % CLEI 2024
      \WorkEntry{2024 \textbf{AelE: a versatile tool for teaching programming and robotics using Arduino
        %\footnote{\url{???}}
      }}
      {Gonzalo Pablo Fernández, Christian Cossio-Mercado}
      {\textbf{50 Conferencia Latinoamericana de Informática (L CLEI 2024), 12-16 de agosto de 2024, Bahía
      Blanca, Argentina}}

      % JAR 2024
      \WorkEntry{2024 \textbf{AelE: una herramienta para la enseñanza de programación basada en Arduino
        %\footnote{\url{???}}
      }}
      {Gonzalo Pablo Fernández, Christian Cossio-Mercado}
      {\textbf{Jornadas Argentinas de Robótica (JAR), 4-7 de junio de 2024, Buenos Aires, Argentina}}

      % JADiCC 2023 PRENDER
      \WorkEntry{2023 \textbf{PRENDER: Una propuesta didáctico-pedagógica para la enseñanza de las Ciencias de la Computación
        %\footnote{\url{???}}
      }}
      {Christian Cossio-Mercado, Gonzalo Pablo Fernández}
      {\textbf{Jornadas Argentinas de Didáctica de las Ciencias de la Computación (JADICC), 1-2 de diciembre de 2023, Neuquén, Argentina}}

      % JADiCC 2023 DOV
      \WorkEntry{2023 \textbf{Enseñanza de fundamentos conceptuales de programación usando Arduino
        %\footnote{\url{???}}
      }}
      {Gonzalo Pablo Fernández, Christian Cossio-Mercado}
      {\textbf{Jornadas Argentinas de Didáctica de las Ciencias de la Computación (JADICC), 1-2 de diciembre de 2023, Neuquén, Argentina}}

      % JAIIO-ex 2023
      \WorkEntry{2023 \textbf{Relevamiento de conocimientos previos de programación en el nivel universitario
        \footnote{\url{https://publicaciones.sadio.org.ar/index.php/EJS/article/view/862/701}}
      }}
      {Gonzalo Pablo Fernández, Cecilia Martínez, Pablo E. ``Fidel'' Martínez López}
      {\textbf{Electronic Journal of SADIO (EJS) 23 (2) 2024 pg 150-175}}

      % JAIIO 2023
      \WorkEntry{2023 \textbf{Relevamiento de conocimientos previos de programación en el nivel universitario
        \footnote{\url{https://publicaciones.sadio.org.ar/index.php/JAIIO/article/view/632/647}}
      }}
      {Gonzalo Pablo Fernández, Cecilia Martínez, Pablo E. ``Fidel'' Martínez López}
      {\textbf{Simposio Argentino de Educación en Informática (SAEI) - Jornadas Argentinas de Informática e Investigación Operativa (JAIIO) 2023}}

      % JADiCC 2022
      \WorkEntry{2022 \textbf{Aprender programación usando bloques y texto en forma simultánea - Un enfoque espiralado
        \footnote{\url{https://repositorio.unne.edu.ar/handle/123456789/50765}}
      }}
      {Gonzalo Pablo Fernández, Pablo E. ``Fidel'' Martínez López, Cecilia Martínez}
      {\textbf{Jornadas Argentinas de Didáctica de las Ciencias de la Computación 2022}}

      % JADiCC 2021
      \WorkEntry{2021 \textbf{Arduino en la Escuela: una herramienta versátil para la enseñanza de programación y robótica
        \footnote{\url{https://jadicc.program.ar/wp-content/uploads/2021/10/JADICC2021_paper_56.pdf}}}}
      {Gonzalo Pablo Fernández, María Belén Ticona Oquendo, Christian Cossio-Mercado}
      {\textbf{Jornadas Argentinas de Didáctica de las Ciencias de la Computación 2021}}

      % ICPR 2020
      \WorkEntry{2020 \textbf{Attribute classification for the analysis of genuineness of facial expressions}}
      {Gonzalo Fernández, María Elena Buemi, Daniel Germán Acevedo, Pablo Negri}
      {\textbf{International Conference on Pattern Recognition (ICPR) 2020}}

      % Tesis
      \WorkEntry{\textbf{Tesis de Licenciatura}}
      {Estimación de la Veracidad de Expresiones Faciales utilizando Aprendizaje Profundo
        %\footnote{\url{https://www.dc.uba.ar/listado-tesis-de-licenciatura/}}
      }
      {Defendida el 13 de Marzo de 2020 \\
      \textbf{Directora}: María Elena Buemi \\
      \textbf{Jurados}: Enrique Segura y Daniel Acevedo}

    \end{itemize}
  }

  \item[b)]{Participación en congresos o acontecimientos nacionales o internacionales
    \begin{itemize}[leftmargin=0.2cm]

      % JADiCC 2024
      \WorkEntry{Integrante del \textbf{Comité de Programa} de las cuartas \textbf{Jornadas Argentinas de Didáctica de las
        Ciencias de la Computación}\footnote{\url{https://jadicc2024.dc.exa.unrc.edu.ar/}}}
      {Universidad Nacional de Río Cuarto}
      {23, 24 y 25 de Octubre de 2024}

      % CLEI 2024
      \WorkEntry{Asistencia a las 50$^a$ \textbf{Conferencia Latinoamericana de Informática}\footnote{\url{https://conferencia2024.clei.org/}}}
      {Universidad Nacional del Sur}
      {14 y 15 de Agosto de 2024}

      % JAIIO 2024
      \WorkEntry{Asistencia al \textbf{Simposio Argentino de Educación en Informática} dentro de las 53$^{\circ}$ \textbf{Jornadas Argentinas de Informática}\footnote{\url{https://jaiio53.clei.org/}}}
      {Universidad Nacional del Sur}
      {14 y 15 de Agosto de 2024}

      % JADiCC 2023
      \WorkEntry{Integrante del \textbf{Comité de Programa} de las terceras \textbf{Jornadas Argentinas de Didáctica de las
        Ciencias de la Computación}\footnote{\url{https://jadicc2023.program.ar/}}}
      {Universidad Nacional del Comahue}
      {1 y 2 de Diciembre de 2023}

      % JIF 2023
      \WorkEntry{Integrante del \textbf{Comité Científico} de las V \textbf{Jornadas de Investigadores en Formación en Ciencia y Tecnología}\footnote{\url{https://sites.google.com/view/jif-cyt-unq-2023/}}}
      {Universidad Nacional de Quilmes}
      {28 y 29 de Septiembre de 2023}

      % JAIIO 2023
      \WorkEntry{Asistencia al \textbf{Simposio Argentino de Educación en Informática} dentro de las 52$^{\circ}$ \textbf{Jornadas Argentinas de Informática}\footnote{\url{https://52jaiio.sadio.org.ar/}}}
      {Universidad Nacional de Tres de Febrero}
      {7 de Septiembre de 2023}

      % JADiCC 2022
      \WorkEntry{Asistencia a las segundas \textbf{Jornadas Argentinas de Didáctica de las
        Ciencias de la Computación}\footnote{\url{https://jadicc2022.unne.edu.ar/}}}
      {Universidad Nacional del Nordeste}
      {18, 19 y 20 de Agosto de 2022}

      % JADiCC 2021
      \WorkEntry{Asistencia a las primeras \textbf{Jornadas Argentinas de Didáctica de las
        Ciencias de la Computación}\footnote{\url{https://jadicc.program.ar/}}}
      {Fundación Sadosky}
      {4, 5 y 6 de Noviembre de 2021}

      % JADiPro 2
      \WorkEntry{Asistencia a las segundas \textbf{Jornadas Argentinas de Didáctica de la
        Programación}\footnote{\url{https://jadipro.unc.edu.ar/}}}
      {Universidad Nacional de Córdoba}
      {7 y 8 de Junio de 2019}
      {Presentación de dos posters:}

      \begin{itemize}
        \item Formación docente complementaria para la implementación de La Programación y su Didáctica\footnote{
          \url{https://drive.google.com/open?id=16m6bdXDUZxmSbBeGUZ23HNGjkllkLbu-}}
        \item Electrónica aplicada para Ciencias y Tecnología con Arduino en la Escuela\footnote{
          \url{https://drive.google.com/open?id=16iX87DDbirIjMgsFyXKE8rtv7wthb7s8}}
      \end{itemize}
      \medskip

      % JADiPro 1
      \WorkEntry{Asistencia a las primeras \textbf{Jornadas Argentinas de Didáctica de la
        Programación}\footnote{\url{https://jadipro.unq.edu.ar/event/1eras-jadipro-2018-05-31-2018-06-01-5/page/introduccion-1eras-jadipro}}}
      {Universidad Nacional de Quilmes}
      {31 de Mayo y 1 de Junio de 2018}
      {Presentación de dos posters:}

      \begin{itemize}
        \item Despertando Vocaciones en Computación por Medio de la Resolución de Problemas con Programación\footnote{
          \url{https://drive.google.com/file/d/11Q6lYX6R9nhxT7xC7PeMLssZSbS47I1a/view}}
        \item La Programación y su Didáctica como parte de la Formación Docente en Ciencia y Tecnología\footnote{
          \url{https://drive.google.com/file/d/1tzRguCk1-sZkQwk8KVUB_AdLvE_K2uGa/view}}
      \end{itemize}
      \medskip

    \end{itemize}
  }

  \item[c)]{Formación de Recursos Humanos.
    \\ No corresponde
  }

  \item[d)]{Participación en Proyectos de Investigación
    \begin{itemize}[leftmargin=0.2cm]

    % UBACyT 2018
    \WorkEntry{\textbf{UBACyT 2018 Mod II GF, Estudiante, Programaci\'on cient\'ifica 2018-2019}}
    {Código 20020170200126BA}
    {Análisis automático de la dinámica facial y corporal}
    {Acreditado y financiado. Director: Daniel Acevedo}

    \end{itemize}
  }

  \item[e)]{Cursos de Posgrado no incluidos en la carrera de Doctorado.
    \\ No corresponde
  }

  \item[f)]{Otros antecedentes científicos no considerados en los puntos anteriores
    \\ No corresponde
  }

\end{enumerate}
