\begin{enumerate}[leftmargin=0.8cm]

  \item[a)]{Titulos Obtenidos

    \begin{itemize}[leftmargin=0.2cm]

      \item {
        \textbf{Licenciatura en Ciencias de la Computación} \\
        En la \emph{Facultad de Ciencias Exactas y Naturales} de la \emph{UBA} el \textbf{25 \slash \ 09 \slash \ 2020}.
        \\ \emph{Título de Tesis}: Estimación de la Veracidad de Expresiones Faciales utilizando Aprendizaje Profundo
          %\footnote{\url{https://www.dc.uba.ar/listado-tesis-de-licenciatura/}}
        \\ \emph{Resumen}: Este trabajo explora y evalúa múltiples métodos de clasificación basados en redes neuronales profundas entrenados para detectar videos que retraten expresiones faciales fingidas.
        \ifdefined\JTP
        \begin{center}
        \textbf{Promedios}

        \begin{tabular}{|r|c|c|}
        \hline
        - & Con Aplazos & Sin Aplazos \\
        \hline
        Con CBC & 8 & 8 \\
        \hline
        Sin CBC & 8,1 & 8,1 \\
        \hline
        \end{tabular}
        \end{center}
        \fi
      }

      \item {
        \textbf{Título Secundario}
        En el \emph{Colegio Paideia} el \textbf{05 \slash \ 03 \slash \ 2009}, \\
        plan \emph{Bachillerato Nacional Bilingüe modalizado en Ciencias y Letras}. \\
        Promedio general: \textbf{7,28}
      }

    \end{itemize}

  }

  \item[b)]{Carrera De Doctorado.

    \textbf{Doctorado en Ciencia y Tecnología - UNQ}

    \begin{itemize}[leftmargin=0.2cm]
        \item[] \emph{Título}: El Gran Bache: La relación entre el enfoque didáctico y el entorno de programación en la enseñanza de la programación
        \item[] \emph{Director de tesis}: Dr. Pablo E. Martínez López
        \item[] \emph{Co-Directora de tesis}: Dra. Cecilia Martínez
        \item[] \emph{Resumen}: Este trabajo busca analizar distintas herramientas y enfoques didácticos para la enseñanza de la programación en el nivel medio y cómo impacatan en el aprendizaje posterior de la programación en el nivel superior, poniendo especial énfasis en la transición entre entornos de programación visuales y lenguajes de programación textuales.
        \item[] \emph{Fecha de ingreso}: Noviembre 2021
    \end{itemize}

  }

  \item[c)]{Becas y distinciones
    \\ No corresponde
  }

  \item[d)]{Tareas de gestión universitaria.

    \begin{itemize}[leftmargin=0.2cm]

      \WorkEntry{Representante suplente por graduados en la \textbf{Comisión de Carrera} (Licenciatura en Ciencias de la Computación)}
      {FCEyN - UBA}
      {Desde Diciembre de 2023.}

    \end{itemize}

  }

  \item[e)]{Otros elementos de juicio que considere valiosos.
    \\ No corresponde
  }

  % \item[d)]{Otros elementos de juicio que considere valiosos
  %
  %   \begin{itemize}[leftmargin=0.2cm]
  %
  %     \WorkEntry{Actualmente cursando el \textbf{Profesorado}}
  %     {}{Desde 2019.}{}
  %
  %     \WorkEntry{Colaboración en la tesis de Ivana L. Nebuloni, graduada de la carrera Diseño de Imagen y Sonido}
  %       {FADU - UBA}
  %       {Durante el segundo cuatrimestre de 2017.}{}
  %
  %     \WorkEntry{Cuarto premio del \textbf{Concurso de Stands 2017 del DC}}
  %       {}
  %       {Diciembre de 2016.}
  %       {Por mi propuesta bajo el nombre \textbf{Piedra, papel o tijera}.}
  %
  %     \WorkEntry{Participación en el \textbf{Torneo Argentino de Programación}\footnote{
  %       \url{http://torneoprogramacion.com.ar/}}\ 2016}
  %       {}
  %       {El día 17 de septiembre de 2016.}
  %       {}
  %       %Puesto: 49 de 79
  %
  %     \WorkEntry{\textbf{Trabajo Práctico final} para la materia \textbf{Organización del Computador II}}
  %     {}{2015}{}
  %
  %   \end{itemize}
  %
  % }

\end{enumerate}

%\input{notas}
