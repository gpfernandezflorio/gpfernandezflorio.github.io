\begin{enumerate}[leftmargin=0.8cm]

  \item[c)]{Formación pedagogica

    \begin{itemize}[leftmargin=0.2cm]

      % Foro ADICRA 2019
      \WorkEntry{Asistencia al \textbf{2$^{\circ}$ Foro de debate ADICRA} 2019\footnote{Organizado por la Asociación de Docentes de Informática y Computación de la República Argentina (\url{http://adicra.org.ar/}). Detalles del evento: \url{http://adicra.com.ar/2-foro-debate-lainformaticacomomateria-sintesis-del-encuentro/}}}
      {ADICRA}
      {El 9 de Noviembre de 2019.}

      % Taller diseño 2018
      \WorkEntry{\textbf{Diseño de Experiencias y Materialización de Divulgación Científica}\footnote{Taller
        organizado por la secretaría de extensión de la facultad, dictado por graduados de FADU para
        extensionistas de Exactas con el objetivo de instruirlos al momento de diseñar material para divulgación.}}
      {FADU - UBA}
      {Entre Agosto y Septiembre de 2018. 15 horas.}

      % Taller para talleristas 2018
      \WorkEntry{\textbf{Seminario de didáctica de las ciencias: Aportes para las intervenciones de
        extensionistas}\footnote{Taller organizado por la secretaría de extensión de la facultad, dictado
        por miembros del CeFIEC para toda la facultad con el objetivo de instruirlos sobre herramientas didácticas
        y pedagógicas utilizadas en las distintas actividades que realizan.}}
      {CeFIEC - FCEyN - UBA}
      {Entre Abril y Mayo de 2018. 12 horas.}

      % Taller para talleristas 2017
      \WorkEntry{\textbf{Trayecto de formación profesional para docentes universitarios en el marco de la preparación
        de talleres para las Semanas de las Ciencias}\footnote{Taller organizado por la secretaría de extensión de la
        facultad, dictado por miembros del CeFIEC para extensionistas de Exactas con el objetivo de instruirlos sobre
        herramientas didácticas y pedagógicas utilizadas al diseñar talleres. Detalles del taller:
        \url{http://cefiec.fcen.uba.ar/cms/index.php/cursos-y-materias-de-posgrado/44-trayecto-de-formacion-para-las-semanas-de-las-ciencias-2017}}}
      {CeFIEC - FCEyN - UBA}
      {Entre Abril y Junio de 2017. 20 horas.}

      % Capacitación profesorados 2017
      \WorkEntry{\textbf{Capacitación sobre Profesorados de Exactas}}
      {CeFIEC - FCEyN - UBA}
      {Durante Marzo de 2017. 2 horas.}

    \end{itemize}
  }

  \item[d)]{Otras actividades docentes

!!!      % Talleres UNQ 2019
      \WorkEntry{Participación en los \textbf{Talleres de Formación Complementaria}}
      {\emph{Colaborador} en los 3 talleres dictados en la \textbf{Universidad Nacional de Quilmes} en el marco del dictado del curso \textbf{La Programación y su Didáctica}.}
      {18 de Mayo de 2019, de 13hs a 17hs y 15 de Junio de 9hs a 17hs.}

!!!      % Intro. a la programación
      \WorkEntry{\textbf{Docente} (con cargo equiparado a \textbf{Ayudante de Segunda}) en el taller
        \textbf{Introducción a la Programación}\footnote{Parte de la iniciativa Program.ar, impulsada por
        la fundación Sadosky, es un taller de dos encuentros de 4 horas cada uno orientado a alumnos de
        secundaria con el objetivo de acercar las Ciencias de la Computación a las escuelas.}}
      {FCEyN - UBA}
      {Durante el segundo cuatrimestre de 2018 y el primero de 2019.}

    \end{itemize}
  }

\end{enumerate}
