\begin{enumerate}[leftmargin=0.8cm]

  \item[a)]{Universitarios

    \begin{itemize}[leftmargin=0.2cm]

      % UNQ
      \WorkEntry{\textbf{Profesor Instructor} en la materia \textbf{Introducción a la Programación}}
      {Universidad Nacional de Quilmes}
      {Desde el primer semestre de 2021.}

      % TC - UNIPE
      \WorkEntry{\textbf{Jefe de Trabajos Prácticos} en la materia \textbf{Teoría de la Computación}}
      {Universidad Pedagógica Nacional}
      {Durante el segundo cuatrimestre de 2023 y desde el segundo cuatrimestre de 2024.}

      % PLP 2024 1C
      \WorkEntry{\textbf{Ayudante de Primera} en la materia \textbf{Paradigmas de Lenguajes de Programación}}
      {FCEyN - UBA}
      {Durante el primer cuatrimestre de 2024.}

      % Unahur
      \WorkEntry{\textbf{Ayudante de Primera} en la materia \textbf{Introducción a la Programación}}
      {Universidad Nacional de Hurlingham}
      {Desde el segundo cuatrimestre de 2021 hasta el primer cuatrimestre de 2022.}

      % Algo 1 2021 1C
      \WorkEntry{\textbf{Ayudante de Primera} en la materia \textbf{Algoritmos y Estructuras de Datos I}}
      {FCEyN - UBA}
      {Durante el primer cuatrimestre de 2021.}

      % Algo 2 2020 2C - 2021 1C
      \WorkEntry{\textbf{Ayudante de Primera} en la materia \textbf{Algoritmos y Estructuras de Datos II}}
      {FCEyN - UBA}
      {Desde el segundo cuatrimestre de 2020 hasta el primer cuatrimestre de 2021.}

      % PLP 2019 2C - 2021 1C
      \WorkEntry{\textbf{Ayudante de Segunda} en la materia \textbf{Paradigmas de Lenguajes de Programación}}
      {FCEyN - UBA}
      {Desde el segundo cuatrimestre de 2019 hasta el primer cuatrimestre de 2020.}

      % LyC 2020 2C
      \WorkEntry{\textbf{Ayudante de Primera} en la materia \textbf{Lógica y Computabilidad}}
      {FCEyN - UBA}
      {Durante el segundo cuatrimestre de 2020.}

      % Sadosky 2019
      \WorkEntry{\textbf{Docente} (ad-honorem) del curso \textbf{La Programación y su Didáctica}\footnote{
        Parte de la iniciativa Program.ar, impulsada por la fundación Sadosky, es un curso orientado a docentes de
        distintos niveles educativos del área informática que propone diferentes e inovadores alternativas para
        enseñar computación. Detalles del curso: \url{http://program.ar/la-programacion-y-su-didactica/}.}}
      {FCEyN - UBA}
      {Durante el segundo cuatrimestre de 2019.}
      %{Dí clases.}

      % Orga I 2019 2C
      \WorkEntry{\textbf{Ayudante de Segunda} en la materia \textbf{Organización del Computador I}}
      {FCEyN - UBA}
      {Durante el segundo cuatrimestre de 2019.}
      %{Dí clases y corregí parciales y talleres.}

      % PLP 2019 2C
      \WorkEntry{\textbf{Ayudante de Segunda} en la materia \textbf{Paradigmas de Lenguajes de Programación}}
      {FCEyN - UBA}
      {Durante el segundo cuatrimestre de 2019.}
      %{Dí clases y corregí parciales y talleres.}

      % Sadosky B2 2019
      \WorkEntry{\textbf{Docente} (equiparado a Ayudante de Segunda) del curso \textbf{La Programación y su Didáctica 2}}
      {FCEyN - UBA}
      {Durante el primer cuatrimestre de 2019.}
      %{Dí clases.}

      % Bases 1C 2019
      \WorkEntry{\textbf{Ayudante de Segunda} en la materia \textbf{Base de Datos}}
      {FCEyN - UBA}
      {Durante el primer cuatrimestre de 2019.}
      %{Dí clases y corregí parciales y trabajos prácticos.}

      % Orga 2 1C 2019
      \WorkEntry{\textbf{Ayudante de Segunda} en la materia \textbf{Organización del Computador II}}
      {FCEyN - UBA}
      {Durante el primer cuatrimestre de 2019.}
      %{Dí clases y corregí parciales y trabajos prácticos.}

      % Algo I 1C 2019
      \WorkEntry{\textbf{Ayudante de Segunda} en la materia \textbf{Algoritmos y Estructuras de Datos I}}
      {FCEyN - UBA}
      {Durante la primera mitad del primer cuatrimestre de 2019.}
      %{Dí clases.}

      % Sadosky 2018
      \WorkEntry{\textbf{Docente} (ad-honorem) del curso \textbf{La Programación y su Didáctica}}
      {FCEyN - UBA}
      {Durante el segundo cuatrimestre de 2018.}
      %{Dí clases.}

      % Sadosky 2017
      \WorkEntry{\textbf{Docente} (ad-honorem) del curso \textbf{La Programación y su Didáctica}}
      {FCEyN - UBA}
      {Durante el segundo cuatrimestre de 2017.}
      %{Armé y dí clases.}

      % TLeng 2017 2C
      \WorkEntry{\textbf{Ayudante de Segunda} en la materia \textbf{Teoría de Lenguajes}}
      {FCEyN - UBA}
      {Durante el segundo cuatrimestre de 2017.}
      %{Dí clases y corregí parciales y trabajos prácticos.}

      % SO 2017-18
      \WorkEntry{\textbf{Ayudante de Primera} en la materia \textbf{Sistemas Operativos}}
      {FCEyN - UBA}
      {Desde el segundo cuatrimestre de 2017 hasta el primer cuatrimestre de 2018.}
      %{Dí clases y corregí parciales, talleres y trabajos prácticos. \\
      %Además, actualicé el \textbf{Taller de Drivers} y el \textbf{Taller de Sistemas de Archivos}.}

      % Algo 3 2017 1C
      \WorkEntry{\textbf{Ayudante de Primera} en la materia \textbf{Algoritmos y Estructuras de Datos III}}
      {FCEyN - UBA}
      {Durante la segunda mitad del primer cuatrimestre de 2017.}
      %{Dí clases y corregí parciales.}

      % LyC 2016 2C
      \WorkEntry{\textbf{Ayudante de Segunda} en la materia \textbf{Lógica y Computabilidad}}
      {FCEyN - UBA}
      {Durante el segundo cuatrimestre de 2016.}
      %{Dí clases y corregí parciales.}

      % Orga II 2016 2C
      \WorkEntry{\textbf{Ayudante de Segunda} en la materia \textbf{Organización del Computador II}}
      {FCEyN - UBA}
      {Durante el segundo cuatrimestre de 2016.}
      %{Dí clases y corregí parciales y trabajos prácticos.}

      % TLeng 2016 1C
      \WorkEntry{\textbf{Ayudante de Segunda} en la materia \textbf{Teoría de Lenguajes}}
      {FCEyN - UBA}
      {Durante el primer cuatrimestre de 2016.}
      %{Dí clases y corregí parciales y trabajos prácticos.}

      % SO 2015
      \WorkEntry{\textbf{Ayudante de Segunda} en la materia \textbf{Sistemas Operativos}}
      {FCEyN - UBA}
      {Durante 2015.}
      %{Dí clases y corregí parciales, talleres y trabajos prácticos.}

    \end{itemize}
  }

  \item[b)]{En otros niveles educativos
    \\ No corresponde
  }

  \item[c)]{Formación pedagogica

    \begin{itemize}[leftmargin=0.2cm]

      % Foro ADICRA 2019
      \WorkEntry{Asistencia al \textbf{2$^{\circ}$ Foro de debate ADICRA} 2019\footnote{Organizado por la Asociación de Docentes de Informática y Computación de la República Argentina (\url{http://adicra.org.ar/}). Detalles del evento: \url{http://adicra.com.ar/2-foro-debate-lainformaticacomomateria-sintesis-del-encuentro/}}}
      {ADICRA}
      {El 9 de Noviembre de 2019.}

      % Taller diseño 2018
      \WorkEntry{\textbf{Diseño de Experiencias y Materialización de Divulgación Científica}\footnote{Taller
        organizado por la secretaría de extensión de la facultad, dictado por graduados de FADU para
        extensionistas de Exactas con el objetivo de instruirlos al momento de diseñar material para divulgación.}}
      {FADU - UBA}
      {Entre Agosto y Septiembre de 2018. 15 horas.}

      % Taller para talleristas 2018
      \WorkEntry{\textbf{Seminario de didáctica de las ciencias: Aportes para las intervenciones de
        extensionistas}\footnote{Taller organizado por la secretaría de extensión de la facultad, dictado
        por miembros del CeFIEC para toda la facultad con el objetivo de instruirlos sobre herramientas didácticas
        y pedagógicas utilizadas en las distintas actividades que realizan.}}
      {CeFIEC - FCEyN - UBA}
      {Entre Abril y Mayo de 2018. 12 horas.}

      % Taller para talleristas 2017
      \WorkEntry{\textbf{Trayecto de formación profesional para docentes universitarios en el marco de la preparación
        de talleres para las Semanas de las Ciencias}\footnote{Taller organizado por la secretaría de extensión de la
        facultad, dictado por miembros del CeFIEC para extensionistas de Exactas con el objetivo de instruirlos sobre
        herramientas didácticas y pedagógicas utilizadas al diseñar talleres. Detalles del taller:
        \url{http://cefiec.fcen.uba.ar/cms/index.php/cursos-y-materias-de-posgrado/44-trayecto-de-formacion-para-las-semanas-de-las-ciencias-2017}}}
      {CeFIEC - FCEyN - UBA}
      {Entre Abril y Junio de 2017. 20 horas.}

      % Capacitación profesorados 2017
      \WorkEntry{\textbf{Capacitación sobre Profesorados de Exactas}}
      {CeFIEC - FCEyN - UBA}
      {Durante Marzo de 2017. 2 horas.}

    \end{itemize}
  }

  \item[d)]{Otras actividades docentes

    \begin{itemize}[leftmargin=0.2cm]

      % SdEC julio 2024
      \WorkEntry{Participación en la \textbf{Semana de la Enseñanza de las Ciencias} 2024}
      {\emph{Docente} en el \textbf{Taller de Didáctica de la Programación}, en el \textbf{Taller de Electrónica Aplicada} y en el \textbf{Taller de Evaluación}.}
      {10, 11 y 12 de Julio de 2024, de 10:30hs a 13:00hs.}

      % Encuentro Profesorados noviembre 2023
      \WorkEntry{Participación en el \textbf{XVII Encuentro Internacional de Profesorados de Enseñanza Superior, Media y Primaria en Ciencias Naturales, Matemática y Tecnología}\footnote{Organizado por el la comisión de profesorados de la facultad (\url{http://www.ccpems.exactas.uba.ar}).
        % Detalles del evento: \url{http://www.ccpems.exactas.uba.ar/cms/index.php/encuentro-profesorados}
      }}
      {\emph{Docente} en el taller \textbf{¿Cómo enseñar a programar? Una didáctica de la programación basada en la resolución de problemas}.}
      {17 de Noviembre de 2023, de 9hs a 11hs.}

      % SdEC julio 2023
      \WorkEntry{Participación en la \textbf{Semana de la Enseñanza de las Ciencias} 2023}
      {\emph{Docente} en el \textbf{Taller de Didáctica de la Programación} y en el \textbf{Taller de Electrónica Aplicada}.}
      {11 y 12 de Julio de 2023, de 10:30hs a 13:00hs.}

      % Encuentro Profesorados noviembre 2022
      \WorkEntry{Participación en el \textbf{XVI Encuentro Internacional de Profesorados de Enseñanza Superior, Media y Primaria en Ciencias Naturales, Matemática y Tecnología}}
      {\emph{Docente} en el \textbf{Taller de Electrónica Aplicada}.}
      {18 de Noviembre de 2022, de 9:30hs a 12hs.}

      % SdEC julio 2022
      \WorkEntry{Participación en la \textbf{Semana de la Enseñanza de las Ciencias} 2022}
      {\emph{Docente} en el \textbf{Taller de Didáctica de la Programación} y en el \textbf{Taller de Electrónica Aplicada}.}
      {12 y 14 de Julio de 2022, de 10hs a 12:30hs.}

      % Ludover 2022
      \WorkEntry{Colaboración en el dictado del \textbf{Taller de Videojuegos Ludover} edición virtual 2022}
      {Fundación Uqbar}{15 encuentros semanales de 3 horas.}

      % FdL mayo 2022
      \WorkEntry{Participación en el \textbf{20$^{\circ}$ Foro Internacional de Enseñanza de Ciencias y Tecnologías}}
      {\emph{Colaborador} en el Taller \textbf{¿Cómo Enseñar a Programar? Una didáctica de la programación basada en la resolución de problemas}.}
      {9 de Mayo de 2022, de 10:30hs a 12:30hs.}

      % Ludover 2021
      \WorkEntry{Colaboración en el dictado del \textbf{Taller de Videojuegos Ludover} edición virtual 2021}
      {Fundación Uqbar}{12 encuentros semanales de 3 horas.}

      % Encuentro Profesorados noviembre 2019
      \WorkEntry{Participación en el \textbf{XIII Encuentro Internacional de Profesorados de Enseñanza Superior, Media y Primaria en Ciencias Naturales, Matemática y Tecnología}}
      {\emph{Docente} en el \textbf{Taller de Programación y Robótica en la escuela}.}
      {22 de Noviembre de 2019, de 9hs a 11:30hs.}
      %{Participé tanto en el dictado como en el armado de la clase.}

      % SdEC julio 2019
      \WorkEntry{Participación en la \textbf{Semana de la Enseñanza de las Ciencias} 2019}
      {\emph{Docente} en el \textbf{Taller de Didáctica de la Programación}, en el \textbf{Taller de Electrónica Aplicada} y
        en el \textbf{Taller de Aplicaciones Móviles para el Aula}.}
      {10, 11 y 12 de Julio de 2019, de 17hs a 19:30hs.}
      %{En los 3 talleres participé tanto en el dictado como en el armado de la clase.}

      % FdL abril 2019
      \WorkEntry{Participación en el \textbf{19$^{\circ}$ Foro Internacional de Enseñanza de Ciencias y Tecnologías}\footnote{
        \url{https://www.el-libro.org.ar/wp-content/uploads/2019/03/web-foro-de-ciencias.pdf}}}
      {\emph{Colaborador} en el Taller \textbf{¿Cómo Enseñar a Programar?}.}
      {29 de Abril de 2019, de 10:30hs a 12:30hs.}

      % Talleres UNQ 2019
      \WorkEntry{Participación en los \textbf{Talleres de Formación Complementaria}}
      {\emph{Colaborador} en los 3 talleres dictados en la \textbf{Universidad Nacional de Quilmes} en el marco del dictado del curso \textbf{La Programación y su Didáctica}.}
      {18 de Mayo de 2019, de 13hs a 17hs y 15 de Junio de 9hs a 17hs.}

      % Encuentro Profesorados noviembre 2018
      \WorkEntry{Participación en el \textbf{XII Encuentro Internacional de Profesorados de Enseñanza Superior,
        Media y Primaria en Ciencias Naturales, Matemática y Tecnología}\footnote{Organizado por el la comisión de
        profesorados de la facultad (\url{http://www.ccpems.exactas.uba.ar}). Detalles del evento:
        \url{http://cefiec.fcen.uba.ar/cms/index.php/12-agenda-de-eventos/48-xii-encuentro-internacional-de-profesorados-de-ensenanza-superior-media-y-primaria-en-ciencias-naturales-y-matematica-2018}}}
      {\emph{Docente} en el \textbf{Taller de Electrónica Aplicada}.}
      {23 de Noviembre de 2018, de 9:30hs a 12hs.}
      %{Participé tanto en el dictado como en el armado de la clase.}

      % Intro. a la programación
      \WorkEntry{\textbf{Docente} (con cargo equiparado a \textbf{Ayudante de Segunda}) en el taller
        \textbf{Introducción a la Programación}\footnote{Parte de la iniciativa Program.ar, impulsada por
        la fundación Sadosky, es un taller de dos encuentros de 4 horas cada uno orientado a alumnos de
        secundaria con el objetivo de acercar las Ciencias de la Computación a las escuelas.}}
      {FCEyN - UBA}
      {Durante el segundo cuatrimestre de 2018 y el primero de 2019.}

      % SdEC julio 2018
      \WorkEntry{Participación en la \textbf{Semana de la Enseñanza de las Ciencias} 2018}
      {\emph{Docente} en el \textbf{Taller de Didáctica de la Programación}, en el \textbf{Taller de Electrónica Aplicada} y
        en el \textbf{Taller de Programación y Robótica}.}
      {10, 11 y 12 de Julio de 2018, de 16hs a 18:30hs.}
      %{En los 3 talleres participé tanto en el dictado como en el armado de la clase.}

      % Encuentro Profesorados noviembre 2017
      \WorkEntry{Participación en el \textbf{XI Encuentro Internacional de Profesorados de Enseñanza Superior,
        Media y Primaria en Ciencias Naturales y Matemática}\footnote{Organizado por el la comisión de
        profesorados de la facultad (\url{http://www.ccpems.exactas.uba.ar}). Detalles del evento:
        \url{http://cefiec.fcen.uba.ar/cms/index.php/12-agenda-de-eventos/45-xi-encuentro-internacional-de-profesorados-de-ensenanza-superior-media-y-primaria-en-ciencias-naturales-y-matematica}}}
      {\emph{Docente} en el \textbf{Taller de Electrónica Aplicada}.}
      {24 de Noviembre de 2017, de 10:30hs a 13hs.}
      %{Participé tanto en el dictado como en el armado de la clase.}

      % SdEC julio 2017
      \WorkEntry{Participación en la \textbf{Semana de la Enseñanza de las Ciencias} 2017}
      {\emph{Docente} en el \textbf{Taller de Electrónica Aplicada} y en el \textbf{Taller de Robótica}.}
      {13 y 14 de Julio de 2017, de 13hs a 15:30hs.}
      %{En ambos talleres participé tanto en el dictado como en el armado de la clase.}

    \end{itemize}
  }

\end{enumerate}
